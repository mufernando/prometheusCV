\datedsubsection{2018 -- present}
	{%
		University of Oregon, United States
    }
	{%
		\textbf{PhD Student}
    }
	{%
        \begin{itemize}
            \item Developed open source population genetics simulation tools within the \href{https://tskit.dev/software/}{tskit} and \href{https://popsim-consortium.github.io/stdpopsim-docs/stable/index.html}{stdpopsim} communities (mostly in Python and C++). 
            \item Analyzed population genomic data and used simulations to tease apart the role of natural selection in shaping genetic variation in the great apes
            \item Developed a machine learning framework based on graph neural networks that takes tree sequences as input to infer evolutionary processes.
        \end{itemize}
        {\footnotesize Advised by Drs. Andrew Kern and Peter Ralph.}
    }

\datedsubsection{2017 -- 2018}
	{%
		University of Wisconsin, United States
    }
	{%
		\textbf{Visiting Researcher}
    }
	{%
        \begin{itemize}
            \item Identified differences in immunity phenotypes between \textit{D. melanogaster} populations.
            \item Analyzed population genomic data to find unusually differentiated immunity genes.
        \end{itemize}
        {\footnotesize Advised by Dr. John Pool.}
    }

\datedsubsection{2016 -- 2018}
	{%
		Universidade de São Paulo, Brasil
    }
	{%
		\textbf{Master's Student}
    }
	{%
        \begin{itemize}
            \item Analyzed a Pool-seq dataset of geographically  and temporally distributed samples.
            \item Modelled the association between spatial and temporal variation in allele frequencies to understand the importance of selection in structuring clinal patterns.
        \end{itemize}
        {\footnotesize Advised by Drs. Rodrigo Cogni and Maria Vibranovski.}
    }

\datedsubsection{2013 -- 2014}
	{%
		Universidade de São Paulo, Brasil
    }
	{%
		\textbf{Undergraduate Researcher}
    }
	{%
        \begin{itemize}
            \item Performed DNA extraction and Sanger sequencing of animal samples.
            \item Analyzed DNA sequence data to build a new phylogeny for \textit{Rhinebothrium}, a genus of tapeworms found in freshwater stingrays.
        \end{itemize}
        {\footnotesize Advised by Dr. Fernando Portella de Luna Marques.}
    }

